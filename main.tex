\documentclass[10pt]{article}
\usepackage[ngerman]{babel}
\usepackage[utf8]{inputenc}
\usepackage[T1]{fontenc}
\usepackage{enumitem}

\begin{document}

\tableofcontents
\newpage

% Zusammenfassung

\begin{center}
\section*{Zusammenfassung von Zero to One}
\addcontentsline{toc}{section}{Zusammenfassung}
\textit{Eine kurze Beschreibung des Inhalts des Buches.}
\end{center}

Diese Zusammenfassung befasst sich mit dem Buch Zero to One von Peter Thiel, einem Mitbegründer von PayPal und Risikokapitalgeber. Thiel argumentiert, dass das Schaffen neuer Dinge sowohl für Unternehmen als auch für den Fortschritt der Menschheit entscheidend ist. Er sieht jedoch eine Stagnation im technologischen Fortschritt und schlägt vor, diese durch die Förderung von Startups zu lösen, die revolutionäre neue Technologien entwickeln.

\subsection*{Vertikaler Fortschritt vs. Horizontaler Fortschritt}
\addcontentsline{toc}{subsection}{Vertikaler Fortschritt vs. Horizontaler Fortschritt}

Thiel unterscheidet zwischen vertikalem (Technologie) und horizontalem (Globalisierung) Fortschritt. Vertikaler Fortschritt schafft Neues, während horizontaler Fortschritt bestehende Dinge verbessert oder vermehrt. Er argumentiert, dass seit den 1970er Jahren der Fortschritt zunehmend horizontal ist, was zu einem intensiveren Wettbewerb um begrenzte Ressourcen führt.

\subsection*{Monopole als Lösung}
\addcontentsline{toc}{subsection}{Monopole als Lösung}

Um diesem Wettbewerb zu entkommen, betont Thiel die Bedeutung von Monopolen. Er behauptet, dass Monopole nicht nur für Unternehmen, sondern auch für die Gesellschaft von Vorteil sind, da sie Unternehmen die Freiheit geben, langfristige Ziele zu verfolgen und sich weniger auf kurzfristige Gewinne zu konzentrieren um die konkurrierenden Unternehmen zu schlagen.

\subsection*{Schaffung neuer Ressourcen durch vertikalen Fortschritt}
\addcontentsline{toc}{subsection}{Schaffung neuer Ressourcen durch vertikalen Fortschritt}

Thiel betont auch die Rolle des vertikalen Fortschritts bei der Schaffung neuer Ressourcen. Während horizontaler Fortschritt zu einem Wettbewerb um vorhandene Ressourcen führt, schafft vertikaler Fortschritt neue Ressourcen durch die Entwicklung neuer Technologien. Dies verringert den Wettbewerb und ermöglicht es, dass mehr Menschen von diesen Ressourcen profitieren.

Insgesamt argumentiert Thiel, dass die Förderung von vertikalem Fortschritt und Monopolen der Schlüssel zur Schaffung einer besseren Zukunft ist. Diese Ideen stehen im Kontrast zu gängigen Annahmen über Wettbewerb und Innovation, aber Thiel glaubt fest daran, dass sie die einzige Möglichkeit sind, um wahre Fortschritte zu erzielen.

\clearpage

%AutorSteckbrief

\begin{center}
\section*{Steckbrief über den Autor}
\addcontentsline{toc}{section}{Steckbrief über den Autor}
\textit{Peter Thiel ist ein einflussreicher Unternehmer, Risikokapitalgeber und Autor, der bekannt ist für seine Gründung von PayPal und seine kontroversen Ansichten in Bezug auf Technologie und Gesellschaft.}
\end{center}

\begin{tabular}{ll}
\textbf{Name:} & Peter Thiel \\
\textbf{Geburtsdatum:} & 11. Oktober 1967 \\
\textbf{Geburtsort:} & Frankfurt am Main, Deutschland \\
\end{tabular}

\subsection*{Überblick}
\addcontentsline{toc}{subsection}{Überblick}
Peter Thiel ist ein deutsch-amerikanischer Unternehmer, Investor und Autor, bekannt für seine Rolle bei der Gründung von PayPal und seine frühe Beteiligung an Facebook. Er ist auch der Gründer von Palantir Technologies und Mitbegründer des Risikokapitalfonds Founders Fund.

\subsection*{Bücher und Ideen:}
\addcontentsline{toc}{subsection}{Bücher und Ideen:}
\begin{itemize}
\item Autor von "Zero to One", in dem er seine Ansichten über Innovation und Unternehmertum teilt.
\item Kontroverse Ansichten zu Technologie, Gesellschaft und Politik.
\end{itemize}

\subsection*{Persönliches}
\addcontentsline{toc}{subsection}{Persönliches}
\begin{itemize}
\item Studium der Philosophie und Jura an der Stanford University.
\item Philanthropische Aktivitäten, darunter das Thiel Fellowship-Programm.
\item Politisch engagiert und Unterstützer von US-Präsident Donald Trump.
\end{itemize}

\clearpage

%BriefPeterThiel

\begin{center}
\section*{Brief an Peter Thiel}
\addcontentsline{toc}{section}{Brief an Peter Thiel}
\textit{Ein Brief mit Fragen an Peter Thiel über seine Erfolge und Ansichten.}
\end{center}

\textbf{Betreff:} Fragen für Peter Thiel

\vspace{10pt}

Sehr geehrter Herr Thiel,

ich hoffe, diese Nachricht erreicht Sie. Mein Name ist Marlon Hinoran und ich habe ein paar Fragen für Sie.

\begin{enumerate}
\item Sie sind Mitbegründer von PayPal und haben eine bedeutende Rolle bei Facebook gespielt. Ist das aufgrund Ihres sozialen Umfelds oder aufgrund Ihres Gespürs, dass Sie systematisch solche Unternehmen noch am Anfang aufspüren?
\item Warum vertreten Sie die Ansicht, dass Wettbewerb nichts Gutes ist und Unternehmen sich daher nicht auf das Wichtige konzentrieren können? Normalerweise wird gesagt, dass Wettbewerb Unternehmen motiviert, im Gegensatz zu Monopolen, in denen ein Unternehmen bereits in der dominierenden Position ist. Was antworten Sie zu dieser Gegenansicht?
\end{enumerate}

Freue mich auf Ihre Antwort.

Mit freundlichen Grüssen,

\vspace{10pt} 
\textit{Marlon Hinoran}

\clearpage

%SiliconValley

\begin{center}
\section*{Silicon Valley das Ende}
\addcontentsline{toc}{section}{Silicon Valley das Ende}
\textit{Eine kurze Geschichte von Silicon Valley und seine Verlagerung nach Texas.}
\end{center}

\subsection*{Der Anfang von Silicon Valley}
\addcontentsline{toc}{subsection}{Der Anfang von Silicon Valley}
Die Geschichte von Silicon Valley beginnt mit William Shockley, einem der Erfinder des Transistors. Er kam nach Palo Alto, weil seine Mutter dort wohnte. Andere Ingenieure folgten ihm, weil es dort günstige Wohnungen gab. Als Shockleys Mitarbeiter ihre eigenen Firmen starteten, blieben sie in der Gegend. So ist Silicon Valley geboren.

\subsection*{Die Vorteile die nicht mehr sind}
\addcontentsline{toc}{subsection}{Die Vorteile die nicht mehr sind}
Der Ort war perfekt für Startups, die kein Geld für teure Büros hatten und einen Ort suchten, an dem sie andere Startup-Gründer treffen konnten und viele Entwickler waren. Heute verlassen aber viele die Bay Area, vor allem wegen den hohen Steuern und Mieten. Das ist ein Problem, weil Silicon Valley von Menschen aus verschiedenen Hintergründen gebaut wurde, nicht nur von reichen Leuten. Heute starte fast niemand mehr in der Silicon Valley eine neue Startup wie damals, da die Entwickler auch von zuhause arbeiten können, es bleiben nur noch grosse Unternehmen wie Google.

\subsection*{Das neue Silicon Valley}
\addcontentsline{toc}{subsection}{Das neue Silicon Valley}
Das neue Silicon Valley ist Texas. Der genaue Ort ist Silicon Hills wo es niedrige Steuern und viel Land gibt das man billig kaufen kann. Im Jahr 2020 haben 35 Unternehmen beschlossen, sich in der Gegend von Austin niederzulassen oder neue Einrichtungen zu eröffnen. Darunter \textbf{Amazon}, \textbf{Facebook}, \textbf{Toyota, Dell} und \textbf{Tesla}.


\clearpage

%KapitelZusammenfassung

\begin{center}
\section*{Kapitel Zusammenfassung}
\addcontentsline{toc}{section}{Kapitel Zusammenfassung}
\textit{Ein Brief mit Fragen an Peter Thiel über seine Erfolge und Ansichten.}
\end{center}

\subsection{Die Herausforderung der Zukunft}

In der Wirtschaft geschieht alles nur einmal, und daher sollte man kein Modell einfach nachahmen. Der Unterschied zwischen Tieren und Menschen liegt in der Technologie. Der Autor argumentiert, dass die Zukunft nicht durch Globalisierung, sondern durch Technologie bestimmt wird, da ohne Technologie China seine Produktion nicht verdoppeln könnte.

\begin{itemize}
\item Horizontaler Fortschritt: Bestehendes kopieren (Globalisierung, Optimierung)
\item Vertikaler Fortschritt: Neues erfinden (Technologie)
\end{itemize}

Die Startup-Mentalität besteht darin, neue Technologien mit einer Gruppe zu erfinden, die weder zu gross noch zu klein ist, um am produktivsten zu sein.

\subsection{Party wie im Jahr 1999}

Der Autor erklärt, dass die 90er Jahre in unserem Glauben die Jahre des Optimismus waren, obwohl die USA Mitte der 90er Jahre in einer Rezession waren. Der erste benutzerfreundliche Browser war der Mosaic-Browser.

Peter Thiel, der damals an PayPal arbeitete, als das Silicon Valley seine besten Zeiten hatte und jede Woche Dutzende neuer Startups gegründet wurden, argumentiert, dass dieses System nicht nachhaltig war.

Er gibt einige Regeln für Startups:

\begin{itemize}
\item Gehe in kleinen Schritten vor.
\item Bleibe schlank und flexibel.
\item Wachse mit der Konkurrenz.
\item Produkte sind wichtiger als ihr Vertrieb.
\end{itemize}

\subsection{Alle glücklichen Unternehmen sind anders}

In Märkten gibt es stets zwei Möglichkeiten: Entweder dominiert eine grosse Firma den Markt, was ein Monopol ist, oder es herrscht Wettbewerb, wodurch der Preis vom Markt bestimmt wird. Monopole können durch verschiedene Mittel erreicht werden:

\begin{itemize}
\item Fragwürdige Taktiken gegen Konkurrenten
\item Erhalt von Lizenzen oder lukrativen Verträgen mit dem Staat
\item Innovation und das Anbieten von etwas Einzigartigem
\end{itemize}

\subsection{Die Ideologie des Wettbewerbs}

Peter Thiel erklärt zu Beginn des Kapitels, dass der Wettbewerb im Gegensatz zu dem, was wir immer gelernt haben, nicht perfekt ist. Nach seiner Ansicht gibt es am Ende für niemanden Gewinne, aber innovative Monopole erwirtschaften Gewinne und schaffen neue Produkte, die der Gesellschaft zugute kommen.

Im Buch werden diese beiden Ansichten zum Wettbewerb dargelegt:

\begin{itemize}
\item \textbf{Marx}: Menschen geraten aufgrund ihrer Lebensumstände in Konflikt, da sie sich voneinander unterscheiden. Arbeiter kämpfen gegen die Bourgeoisie, weil sie widersprüchliche Ziele und Ideen haben.
\item \textbf{Shakespeare}: Die Menschen sind gleich. Sie haben nicht viele Gründe zu kämpfen, tun es aber dennoch. Je mehr sie jedoch gegeneinander kämpfen, desto ähnlicher werden sie einander.
\end{itemize}

\subsection{Die letzten werden die ersten sein}

Es gibt zwei wichtige Zeiträume in der Entwicklung eines entstehenden Marktes, um ein wirksames Monopol zu schaffen; der erste Zug und der letzte Zug.

Thiel legt grosses Gewicht auf das Verständnis des richtigen Unternehmensbewertungsprozesses (und wie man gültige Bewertungen im Vergleich zu aufgeblähten Bewertungen unterscheidet).

Die Gleichung für den Wert eines Unternehmens heute ist der Wert der zukünftigen Gewinne. Hier ist ein Beispiel unter Verwendung der vorherigen Definition.

\begin{center}
\texttt{Wert des Unternehmens heute = Wert der zukünftigen Gewinne}
\end{center}

\subsection{Das Leben ist kein Glücksspiel}

Thiel argumentiert, dass man sich nicht darauf konzentrieren sollte, vielseitig zu sein, sondern seine ganze Energie auf eine Fähigkeit konzentrieren sollte, um in diesem Bereich der Beste zu sein und zu den Top 1 Prozent zu gehören. Im Gegensatz zur Schule, in der man lernt, überall gut zu sein, um einen guten Notendurchschnitt zu erzielen. Thiel betont, dass es wichtiger ist, in einem Bereich herausragend zu sein, als in vielen Bereichen durchschnittlich zu sein.

\subsection{Die Spur des Geldes}

Die 80/20-Regel besagt:

\begin{itemize}
\item 80 Prozent der Ergebnisse aus 20 Prozent der Anstrengungen kommen. Dies gilt auch für den Erfolg von Start-ups.
\end{itemize}

Wenn man sich ansieht, wo das Geld hinfliesst, wird deutlich, dass nur wenige Unternehmen wirklich erfolgreich sind. Die meisten Start-ups scheitern, während einige gerade genug Geld verdienen, um über die Runden zu kommen. Nur eine kleine Minderheit erzielt exponentielles Wachstum. Deshalb konzentrieren sich Venture Capitalists auf diese wenigen Unternehmen, die das Potenzial haben, grosse Gewinne zu erzielen.

\subsection{Geheimnisse}

Er betont die Bedeutung von Geheimnissen, die darauf warten entdeckt zu werden, um ein erfolgreiches Unternehmen aufzubauen. Er verwendet HP als Beispiel dafür, was passiert, wenn ein Unternehmen aufhört, an Geheimnisse zu glauben und nach ihnen zu suchen. Jedes grossartige Unternehmen hat ein Geheimnis, wie z.B. Googles PageRank-Algorithmus.

\subsection{Grundlagen}

Es ist wichtig, dass die Persönlichkeiten und Fähigkeiten der Gründer sich ergänzen.

Bei der Gründung eines Unternehmens gibt es drei wichtige Aspekte:

\begin{itemize}
\item \textbf{Besitz:} Wer das Unternehmen rechtlich besitzt
\item \textbf{Besitz:} Wer das Unternehmen tatsächlich führt
\item \textbf{Kontrolle:} Wer die Angelegenheiten des Unternehmens formell regelt
\end{itemize}

Thiel bevorzugt einen kleinen Vorstand und betont, dass jeder im Unternehmen entweder "im Bus" oder "ausserhalb des Busses" ist. Eine Person, die ein Gehalt über den positiven Beitrag zum Wachstum des Unternehmens stellt, ist eine Person, die versucht, Werte zu extrahieren, anstatt Werte zu schaffen.

Das Ziel eines Unternehmens sollte es sein, die richtige Menge an Eigenkapital zu geben, um langfristiges Denken zu fördern. Werte für die Zukunft aufzubauen, sowie die richtige Menge an Bargeldgehalt, um die unmittelbaren Bedürfnisse zu befriedigen.

\subsection{Die Mechanik der Mafia}

Es wird erklärt wie man die Mafia als Modell für den Aufbau eines erfolgreichen Unternehmens betrachten kann. Er spricht über das Finden von Mitarbeitern, die Schaffung einer gemeinsamen Kultur, die Konzentration auf eine bestimmte Sache und den Aufbau einer starken Anhängerschaft. Thiel verwendet diese Ideen, um zu zeigen, wie man ein Unternehmen gründet und es erfolgreich macht.

\subsection{Wenn du es baust, kommen sie dann?}

Für Startups ist es genauso wichtig, sich um den Verkauf zu kümmern wie um das Produkt. Der Vertrieb ist oft der Engpass für ein Geschäft.

\textbf{Grundlagen zum Verkauf eines Produkts:}

\begin{itemize}
\item Der Wert eines Kunden über seine Lebensdauer ist wichtiger als die Kosten, um ihn zu gewinnen.
\item Je teurer das Produkt ist, desto mehr kostet es, es zu verkaufen.
\item Wenn du keinen Verkäufer siehst, bist du wahrscheinlich selbst der Verkäufer.
\end{itemize}

\subsection{Mensch und Maschine}

Mensch-Maschine-Teamarbeit vs. Roboter nehmen unsere Jobs weg.
Globalisierung = teilweise Ersatz, Technologie = Ergänzung.

\textbf{Beispiel Igor (PayPal):}
PayPal verlor Mitte der 2000er Jahre monatlich 10 Millionen Dollars durch Kreditkartenbetrug.
Ein Team von Mathematikern entwickelte einen Algorithmus, um das Problem zu lösen.
Betrüger passten sich schnell an, daher wurde ein Team von Analysten eingesetzt, um verdächtige Transaktionen zu überprüfen.
Diese Mischung aus menschlicher und künstlicher Intelligenz, genannt "Igor", führte dazu, dass PayPal im Jahr 2002 erstmals Gewinn machte.
Diese Methode wurde auch vom FBI genutzt und inspirierte später die Gründung von Palantir.

\subsection{Grüne Perspektive}

Die meisten Cleantech-Unternehmen scheiterten, weil sie die sieben grundlegenden Fragen nicht beantworten konnten:

\begin{enumerate}
\item \textbf{Technologie:} Schaffen Sie bahnbrechende Innovationen?
\item \textbf{Timing:} Ist es der richtige Zeitpunkt?
\item \textbf{Monopol:} Beginnen Sie mit einem kleinen Marktanteil?
\item \textbf{Team:} Haben Sie das richtige Team?
\item \textbf{Vertrieb:} Können Sie Ihr Produkt liefern?
\item \textbf{Beständigkeit:} Ist Ihre Position langfristig verteidigbar?
\item \textbf{Geheimnis:} Haben Sie eine einzigartige Chance erkannt?
\end{enumerate}

\subsection{Das Paradox der Gründer}

Gründer sind wichtig, weil sie das Beste aus allen Mitarbeitern herausholen können, nicht nur weil ihre Arbeit wertvoll ist.
Es geht darum, wie Menschen und Gründer unterschiedliche Eigenschaften haben.
Die Lektion für Gründer ist, dass sie plötzlich von Bewunderung zu Kritik wechseln können.
Die grösste Gefahr für Gründer ist, zu sehr an ihren eigenen Erfolg zu glauben und den Verstand zu verlieren. Aber es ist auch gefährlich, den Glauben an den Erfolg zu verlieren und Weisheit mit Enttäuschung zu verwechseln.

\clearpage

%Zusatzaufgaben

\begin{center}
    \section*{Zusatzaufgaben}
    \addcontentsline{toc}{section}{Zusatzaufgaben}
    \textit{Eine Sammlung von Briefen, Analysen und Geschichten}
\end{center}

\subsection*{Brief an Elon Musk}
\addcontentsline{toc}{subsection}{Brief an Elon Musk}

\begin{description}
    \item[Auftrag: 10]
\end{description}

\begin{quote}
    \textbf{Betreff:} Fragen für Elon Musk
    
    Sehr geehrter Herr Musk,
    
    ich hoffe, diese Nachricht erreicht Sie. Mein Name ist Marlon Hinoran und ich habe ein paar Fragen für Sie.
    
    \begin{enumerate}
        \item War die Demonstration des Cybertrucks mit der eingeschlagenen Scheibe absichtlich?
        \item Sie sind dafür bekannt, in bestehende oder neue Märkte einzutreten und diese zu revolutionieren, sei es mit PayPal, Tesla oder SpaceX. Denken Sie, dass Sie dasselbe in einem sehr konkurrenzbetonten Markt wie dem Markt für Smartphones erreichen könnten?
        \item Sie haben viel mit dem Gericht zu kämpfen, wie es bei vermögenden und prominenten Personen üblich ist, sei es von Privatpersonen oder Unternehmen. Wie bewältigen Sie den Stress, der mit solchen Ereignissen verbunden ist?
        \item Wann und warum würden Sie den Gang herunterfahren und in den Ruhestand gehen?
        \item Wenn Sie eine wichtige Entscheidung treffen müssen, welchen Prozess verwenden Sie? Zum Beispiel beim Kauf von Twitter.
    \end{enumerate}
    
    Freue mich auf Ihre Antwort.
    
    Mit freundlichen Grüßen,
    Marlon Hinoran
\end{quote}

\subsection*{Das nächste PayPal}
\addcontentsline{toc}{subsection}{Das nächste PayPal}

\begin{description}
    \item[Auftrag: 11]
\end{description}

PayPal wurde eingeführt, um den Geldversand zu vereinfachen. Wenn wir uns jetzt fragen, was möglicherweise das nächste große Zahlungsmittel sein wird, sehen wir, dass die einzigen Revolutionen im Finanzbereich in letzter Zeit die Kryptowährungen sind. Das bedeutet, dass das nächste PayPal nicht mehr Dollar oder Euro verschicken könnte, sondern Bitcoin, Ethereum und ähnliche. Ein Problem besteht darin, dass Kryptowährungen bisher nicht weit verbreitet im Alltag genutzt werden, sondern vor allem zum Investieren oder Spekulieren verwendet werden.

\subsection*{Wie verdient Peter Thiel sein Geld in 2024}
\addcontentsline{toc}{subsection}{Wie verdient Peter Thiel sein Geld in 2024}

\begin{description}
    \item[Auftrag: 15]
\end{description}

Seit März 2020 ist Thiel an dem jungen Biotech-Unternehmen AbCellera beteiligt und sitzt mittlerweile im Aufsichtsrat. Er kündigte jedoch dieses Jahr (2024) aus persönlichen Gründen seinen Rückzug aus dem Aufsichtsrat an.

Seitdem ist nicht ganz klar, was sein neues großes Projekt ist. Die letzte bekannte Nachricht über ihn besagt, dass er in eine Organisation namens "Enhanced Games" investiert hat, die eine Olympiade schaffen möchte, welche die Verwendung von leistungssteigernden Drogen nicht verbietet, sondern sogar fördert.

Das Hauptargument der Organisation ist, dass Technologien sich weiterentwickeln und dass man Menschen nicht davon abhalten sollte, sich zu optimieren. Zudem argumentiert die Organisation, dass es möglicherweise Steroide oder Medikamente in der Zukunft geben wird, die den Menschen optimieren könnten, die jedoch derzeit nicht nachweisbar sind.

\subsection*{Die Geschichte des Buches „Zero To One“}
\addcontentsline{toc}{subsection}{Die Geschichte des Buches „Zero To One“}

\begin{description}
    \item[Selbst gefundenen Auftrag]
\end{description}

Der Ursprung von "Zero to One" kommt von einem Kurs, von Peter Thiel im Jahr 2012 an der Stanford University. Ein Student namens Blake Masters machte Notizen während dem Kurs. Thiel und Masters arbeiteten dann zusammen, um diese Notizen zu überarbeiten und richtiges Buch erstellen.

\subsection*{PayPal die Geschichte}
\addcontentsline{toc}{subsection}{PayPal eine Geschichte}

\begin{description}
    \item[Auftrag: 15]
\end{description}


\begin{itemize}[label=\textbullet]
    \item \textbf{Dezember 1998}: Gründung von Confinity durch Max Levchin, Peter Thiel, Luke Nosek, Yu Pan, Ken Howery und Russel Simmons.    
    \item \textbf{Januar 1999}: Start von X.com durch Elon Musk.    
    \item \textbf{Juli 1999}: Ankündigung von PayPal als Online-Zahlungsservice durch Confinity.    
    \item \textbf{September 1999}: PayPal wird der Öffentlichkeit zugänglich gemacht.   
    \item \textbf{Dezember 1999}: Confinity startet die voll funktionsfähige PayPal-Website.    
    \item \textbf{März 2000}: Ankündigung von Billpoint durch eBay als Konkurrenz zu PayPal.    
    \item \textbf{März 2000}: Fusion von Confinity und X.com, wobei Elon Musk als CEO bleibt.    
    \item \textbf{April 2000}: Elon Musk wird CEO von PayPal.    
    \item \textbf{Juni 2000}: Einführung von kostenpflichtigen PayPal-Konten mit Betrugsschutz.    
    \item \textbf{Februar 2002}: PayPal geht an die Börse.    
    \item \textbf{Juli 2002}: Übernahme von PayPal durch eBay.    
    \item \textbf{Januar 2008}: PayPal übernimmt Fraud Sciences.    
    \item \textbf{November 2008}: PayPal erwirbt Bill Me Later.    
    \item \textbf{August 2009}: Start von Student Accounts durch PayPal.    
    \item \textbf{November 2009}: Öffnung der PayPal-Plattform für andere Dienste.    
    \item \textbf{Juni 2011}: PayPal erreicht 100 Millionen Nutzer.    
    \item \textbf{September 2013}: PayPal übernimmt Braintree.    
    \item \textbf{September 2014}: PayPal ermöglicht Bitcoin-Transaktionen.    
    \item \textbf{September 2014}: Ankündigung der Abspaltung von PayPal von eBay.    
    \item \textbf{März 2015}: PayPal übernimmt Paydiant.    
    \item \textbf{Juli 2015}: PayPal kauft Xoom.    
    \item \textbf{November 12, 2015}: PayPal tritt erstmals im Juni 2016 in die Fortune 500 ein.    
    \item \textbf{Januar 2016}: Venmo verarbeitet im Laufe dieses Monats 1 Milliarde US-Dollar an Zahlungen.    
    \item \textbf{Februar 2016}: PayPal und Braintree starten PayPal Commerce.    
    \item \textbf{Februar 2016}: PayPal führt eine Neugestaltung der PayPal-App ein.    
    \item \textbf{Februar 2017}: PayPal startet einen Slack-Bot für peer-to-peer-Zahlungen mit dem Code /PayPal und dem Slack-Benutzernamen.
    \item \textbf{Juli 2017}: PayPal kündigt das Global Sellers Program in Partnerschaft mit Webinterpret an, um Online-Händlern Lösungen für den grenzüberschreitenden Handel anzubieten.

\end{itemize}






\end{document}